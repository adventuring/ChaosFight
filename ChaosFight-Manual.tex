\documentclass[10pt,twocolumn]{article}
\usepackage[margin=0.75in]{geometry}
\usepackage{graphicx}
\usepackage{enumitem}
\usepackage{fancyhdr}

\title{ChaosFight\\Atari 2600 Game Manual}
\author{Interworldly Adventuring, LLC}
\date{October 2025}

\pagestyle{fancy}
\fancyhf{}
\fancyhead[C]{ChaosFight Manual}
\fancyfoot[C]{\thepage}

\begin{document}

\maketitle

\begin{abstract}
ChaosFight is an exciting multiplayer fighting game for the Atari 2600. Battle against friends or AI opponents in side-view combat arenas with unique character abilities and physics-based gameplay.
\end{abstract}

\section{Game Overview}

ChaosFight is a side-view fighting game where up to 4 players can battle simultaneously. Players select from 16 unique characters, each with different attack types (melee or ranged) and special abilities.

\subsection{Game Modes}
\begin{itemize}
\item \textbf{1-Player Mode}: Battle against an AI opponent
\item \textbf{2-Player Mode}: Human vs Human combat
\item \textbf{4-Player Mode}: Full multiplayer with Quadtari adapter (requires additional hardware)
\end{itemize}

\subsection{Character Selection}
\begin{enumerate}
\item Use the joystick to cycle through 16 available characters
\item Press the fire button to lock in your selection
\item Move the joystick again to unlock and change your choice
\item Locked characters are indicated by colored borders
\end{enumerate}

Each character has unique stats:
\begin{itemize}
\item \textbf{Attack Type}: Melee (close-range) or Ranged (projectile attacks)
\item \textbf{Damage}: Base damage dealt per attack
\item \textbf{Special Abilities}: Some characters have enhanced health or special moves
\end{itemize}

\subsection{Gameplay Controls}
\begin{itemize}
\item \textbf{Joystick Left/Right}: Move character horizontally
\item \textbf{Joystick Up}: Jump
\item \textbf{Joystick Down}: Guard (reduces incoming damage)
\item \textbf{Fire Button}: Attack
\end{itemize}

\subsection{Combat System}
\begin{itemize}
\item \textbf{Health}: Each character starts with 100 health points
\item \textbf{Damage}: Attacks deal 15-40 damage depending on character and attack type
\item \textbf{Recovery}: After taking damage, characters enter a brief recovery period where they cannot be damaged again
\item \textbf{Knockback}: Successful attacks push opponents back
\end{itemize}

\subsection{Attack Types}
\begin{itemize}
\item \textbf{Melee Attacks}: Close-range attacks that hit adjacent opponents
\item \textbf{Ranged Attacks}: Projectile attacks that can hit distant targets
\end{itemize}

\subsection{Physics and Movement}
Characters have realistic physics with:
\begin{itemize}
\item \textbf{Gravity}: Characters fall naturally when jumping
\item \textbf{Momentum}: Movement builds and decays naturally
\item \textbf{Collision}: Characters cannot pass through each other or solid objects
\end{itemize}

\section{Game Arenas}
Choose from multiple battle arenas, each with unique layouts and obstacles.

\subsection{Arena 1: Basic Platform}
A simple arena with platforms on the left and right sides.

\subsection{Arena 2: Central Platform}
Features a central elevated platform with side platforms.

\section{Console Switches}

\subsection{Game Reset}
Returns to the character selection screen.

\subsection{Color/B\&W Switch}
Toggles between color and monochrome display modes. In B\&W mode, characters maintain their solid colors but with adjusted brightness.

\subsection{Game Select}
Toggles pause mode. When paused, the game freezes and displays a black background with characters visible.

\section{Winning the Game}
The last player with health remaining wins the match. The winning character is displayed in the center of the screen with their player number.

\section{Technical Information}
\begin{itemize}
\item \textbf{Platform}: Atari 2600
\item \textbf{Memory}: 64K bank-switched cartridge
\item \textbf{Players}: 1-4 players
\item \textbf{Controllers}: Standard Atari joysticks
\item \textbf{Quadtari Support}: Required for 4-player mode
\end{itemize}

\section{Tips and Strategies}
\begin{itemize}
\item Master your character's attack timing
\item Use the environment to your advantage
\item Time your jumps to avoid attacks
\item Guard to reduce incoming damage
\item Watch your opponent's recovery periods
\end{itemize}

\end{document}
